\chapter{Conhecendo novas terras}

Como diz um provérbio tailandês, ``Se você não deixar seu lar, não vai
conhecer a estrada, não vai desenvolver habilidades ou ganhar
sabedoria.'' Ajahn Chah conhecia bem esse ditado que o fez perceber que
na sua terra natal seria difícil encontrar professores capacitados.
Portanto, após passar no exame do Nak Thamm Tri, decidiu sair em busca
de conhecimento em outras localidades. Em 1941 deixou Wat Ban Kó Nók e
foi para Wat Suan Sawan, em Pibun Mangsahan, mas uma vez que aquele
monastério ainda não possuía uma escola de Dhamma, Luang Pó
matriculou-se na escola de Wat Po Tak, que era perto o suficiente para
que pudesse estudar lá e voltar para dormir em Wat Suan Sawan após as
aulas.

Em Wat Suan Sawan havia apenas duas cabanas e um templo, e ambos estavam
lotados de monges e noviços. Até mesmo tropas do exército às vezes
passavam a noite lá, pois ainda se encontravam no período da Grande
Guerra da Ásia Oriental. Por haver tantos monges vivendo juntos, comida
era escassa. Os alimentos obtidos em \emph{pindapāta}\footnote{O ato de
  os monges saírem pelas ruas recolhendo alimentos como esmola (pāli).}
não eram suficientes porque o vilarejo onde os monges iam recolher
esmolas possuía apenas algumas casas. A água potável tinha que ser
tirada de um poço localizado a cerca de um quilômetro do monastério, mas
água para limpeza podia ser obtida do rio Mun, que passava ali perto.

Em 1942, após estudar as escrituras budistas naquele monastério por um
ano, Luang Pó ainda se sentia insatisfeito. Ele deixou Wat Suan Sawan e
caminhou até Wat Ban Nong Lak, em Muang Sam Sip, onde Prah Kru Antatham
Vijan era o abade. Sendo a estação seca quando Luang Pó chegou, a comida
era muito pobre e seu amigo de viagem reclamou, sugerindo que seguissem
adiante. Luang Pó gostou do professor em Wat Nong Lak mas não queria
desagradar seu amigo, então ambos mudaram-se para Wat Ban Keng Yay, em
Nat Jarên, onde Ajahn Mahā Jéng era abade. Passado um ano, e tendo
ganhado a graduação Nak Thamm To, Luang Pó decidiu que já havia
permanecido ali tempo suficiente e partiu em 1943 para estudar novamente
com Prah Kru Antatham Vijan em Wat Ban Nong Lak.

Naquele ano, Luang Pó focou todas suas energias e interesse em estudar
as escrituras. Lá, estudou para o exame de Nak Thamm Ek e também
gramática \emph{pāli,} porque ele gostava muito do método de estudo
daquele monastério. Ele estava bastante esperançoso de que o resultado
de seu exame ao final do ano fosse satisfatório, se esquecendo que a
impermanência do mundo continuava trabalhando por trás das cenas e só
esperava uma chance para mais uma vez o surpreender.

Ao fim do vassa, Luang Pó recebeu a informação de que seu pai estava
muito doente e essa notícia o fez ficar indeciso entre continuar seus
estudos ou cuidar de seu pai. No final decidiu que seu pai era uma
pessoa para quem ele tinha um grande débito de gratidão e que era seu
dever tentar pagar um pouco esse débito de qualquer maneira possível. No
que diz respeito aos estudos, refletiu que, se não morresse antes, ainda
teria a chance de retomá-los. Luang Pó deixou todos os livros e exames
para trás e apressou-se em voltar para casa para visitar seu pai e
ajudar a cuidar dele tanto quanto lhe fosse possível. Ao chegar,
descobriu que a saúde de seu pai vinha se deteriorando progressivamente
e seu estado era preocupante.

Desde que Luang Pó se tornou monge, seus pais estavam sempre muito
orgulhosos, pois viam que seu filho era sincero em sua ordenação e em
seus estudos do Dhamma. Sempre que Luang Pó tinha a oportunidade de
visitá-los, seu pai gostava de falar sobre a diferença entre a vida
monástica e a vida laica e sempre pedia a seu filho: ``Não abandone o
manto monástico! Viver como monge é bom, se você deixar o manto você só
vai encontrar confusão e dificuldades sem fim.'', e sempre que Luang Pó
ouvia esse pedido ele permanecia em silêncio, sem dar uma resposta
definitiva. Mas desta vez, quando seu pai enfermo lhe fez o mesmo pedido
-- talvez pela última vez -- Luang Pó lhe deu uma resposta que o deixou
muito satisfeito: ``Não vou deixar o manto, por que faria isso?''

Além de desejar que o filho continuasse na vida monástica, seu pai
também se preocupava com seus estudos de Nak Thamm mais do se preocupava
com sua própria saúde. Quando soube que o exame final seria em apenas
alguns dias, pediu ao filho que voltasse ao monastério e não
desperdiçasse aquela oportunidade. Porém, após refletir sobre o estado
de saúde de seu pai, Luang Pó decidiu ficar até a última hora daquele
que lhe havia dado a vida. No total, Ajahn Chah passou treze dias em
casa ajudando a cuidar de seu pai com completa dedicação até seu último
momento.

Durante todo o período em que cuidava de seu pai enfermo, até o momento
da sua morte, Luang Pó contemplava o mundo material. Ele contemplava
como todos os fenômenos condicionados (\emph{sankhāra}) surgem, se
desenvolvem e se desfazem. Surgiu desencanto em seu coração: ``Isso é
tudo o que a vida tem a oferecer? Tanto o rico como o pobre correm em
direção à morte, que é o destino final da vida de todos. Velhice, doença
e morte são a herança que todos têm que receber, não importa se a
queiram ou não -- não se vê quem consiga escapar.''

Encerrado o funeral, Luang Pó voltou para Wat Ban Nong Lak e continuou
seus estudos. Às vezes se lembrava da imagem de seu pai deitado e
enfermo, seu corpo emaciado e enfraquecido, se lembrava do pedido que
ele lhe havia feito e da imagem dele morrendo bem em frente a seus olhos
e tudo isso lhe fazia sentir ainda mais desencanto pelo mundo. Esse
sentimento retornava frequentemente e o fez decidir-se firmemente a
dedicar esta vida à prática do Dhamma, a alcançar o fim de todo
sofrimento ainda nesta vida. Tudo isso o levou a fazer um voto, tal como
relatou em certa ocasião:

``Antes de decidir me dedicar plenamente à prática do Dhamma, eu pensei:
`O \emph{Buddha} \emph{Sāsanā} está presente no mundo, mas por que
algumas pessoas praticam e outras não? Alguns praticam só um pouco e
logo desistem, ou, quando não desistem, não praticam com dedicação
total. Por quê? É porque eles não sabem\ldots{}' Eu então tomei uma
resolução no meu coração: `Muito bem, nesta vida vou sacrificar este
corpo e mente, deixá-los morrer uma vida que seja, vou praticar de
acordo com os ensinamentos do Buddha em todos os detalhes. Vou praticar
até alcançar conhecimento ainda nesta vida, porque se não alcançá-lo vou
ter que voltar e sofrer novamente. Vou renunciar a tudo, vou me esforçar
para praticar, não importa quão sofrido seja, quão difícil seja, vou
viver esta vida como se fosse o passar de um dia e uma noite, vou
jogá-la fora. Vou praticar de acordo com os ensinamentos do Buddha, com
o Dhamma, até compreender por que este \emph{samsāra} é tão difícil e
complicado\ldots{}''

Naquele mesmo ano, Luang Pó começou a traduzir o Dhammapada do pāli para
o tailandês porque esta era uma exigência para obter a graduação de
``Prinha Tri''. Começou também a praticar meditação, mas no início as
coisas não foram muito bem, como contou mais tarde a seus discípulos:

``Não consegui nada no primeiro ano de minha prática, eu só `meditava'
sobre coisas que queria comer, era só confusão! Era muito ruim.
Sentava-me e às vezes era como se realmente estivesse comendo bananas,
eu sentia como se estivesse quebrando a banana e colocando-a em minha
boca. Isso acontece assim mesmo, tudo isso é assunto para a prática --
mas não tenha medo! Isso vem sendo dessa forma já faz muitas vidas, e é
por isso que, quando nos decidimos a praticar, é tudo tão difícil e
complicado.''

Ao mesmo tempo em que a prática de Luang Pó caminhava, seu velho
inimigo, o desejo sexual, continuava retornando para desafiá-lo e
colocar em teste sua resolução de trilhar o caminho do Dhamma. Numa
certa ocasião, o desejo sexual o fez seguir uma linha de raciocínio
equivocada que quase lhe custou sua ordenação monástica:

``\ldots{} eu mesmo certa vez pensei nisso (abandonar a vida monástica).
Quando tinha apenas cinco ou seis anos como monge, eu pensei no Buddha:
ele praticou por cinco ou seis anos e alcançou a iluminação, mas eu
ainda estava apegado a coisas mundanas, eu ainda desejava voltar para
elas. `Talvez eu devesse ir experimentar o mundano um pouco, então terei
melhor compreensão. Mesmo o Buddha teve Rāhula\footnote{Siddhattha
  Gotama, antes de se tornar monge e alcançar a iluminação, era casado e
  teve um filho chamado Rāhula.}, talvez esteja sendo rigoroso demais
comigo mesmo\ldots{}' Eu fui meditando sobre isso e sabedoria surgiu em
mim: `Tudo bem, mas o que temo é que este `Buddha'\footnote{Na
  Tailândia, especialmente em áreas rurais, é comum se referirem aos
  monges como `Buddha', por exemplo, uma pessoa pode dizer: `Hoje havia
  cinco Buddhas no templo', se referindo a quantos monges estavam lá.}
aqui não vai ser tão bom quanto aquele lá. Temo que este aqui vá acabar
se afundando na lama; é pouco provável que ele vá conseguir sair da lama
como aquele conseguiu\ldots{}' Sabedoria surgiu dessa forma e fez
oposição às \emph{kilesas}.''

Mesmo enfrentando essas dificuldades, Luang Pó seguia em frente com seus
estudos e, tendo terminado de traduzir várias seções do Dhammapada,
começou a refletir e comparar seu comportamento com o comportamento dos
monges na época do Buddha e concluiu que eram muito diferentes. Ele
começou a sentir-se decepcionado com os estudos escolásticos, pois não
lhe parecia que esse pudesse ser o caminho para transcender
\emph{dukkha}, ou que o Buddha aprovaria uma pessoa se ordenar tendo
esse tipo de estudo como único objetivo. Ele considerou aprender um
pouco mais sobre a prática do Dhamma para poder saber quão diferentes
esses dois caminhos eram, mas não conseguia encontrar um professor que
julgasse ter qualificação suficiente para ensinar meditação no local
onde estava. Por isso, decidiu voltar a seu monastério original, Wat Ban
Kó Nók.

Durante a estação seca de 1945, Luang Pó ouviu notícias de que havia um
professor de meditação em Udon Thani e foi até lá para aprender o modo
de prática por um período curto, mas achou os ensinamentos
insatisfatórios e então voltou a Wat Ban Kó Nók para passar o
\emph{vassa}. Naquele \emph{vassa} ele teve oportunidade de pagar seu
débito de gratidão com seu antigo professor ajudando a dividir o fardo
de ensinar as escrituras aos monges mais jovens. Porém, durante as
aulas, ele notava que aqueles monges e noviços não eram tão sinceros em
seus estudos como ele gostaria que fossem. Alguns eram desrespeitosos,
outros estudavam sem interesse algum, outros eram preguiçosos, e tudo
isso o fez se sentir ainda mais desanimado com o modo de vida de monges
que não se dedicavam à prática do Dhamma.

Ao mesmo tempo em que ensinava Nak Thamm aos demais, Luang Pó também
estudava para o exame de Nak Thamm Ek, e obteve a graduação. Logo após o
fim do \emph{vassa,} preparou acessórios para a vida na estrada e
começou sua viagem a pé em busca do caminho de \emph{kammatthāna} --
aquele que ele decidira seguir a partir de então.
